\documentclass{article}
\usepackage[utf8]{inputenc}

\title{Scoping Exercise 1- Jack Mathieson}
\author{jack.mathieson }
\date{August 2019}

\begin{document}

\maketitle
\section{FOAR705 Proof of Concept (for past ethnographic research project and current)}
\textbf{
\subsection{Difficult Jobs in the Ethnographic process:}
}

1 \textbf{Finding relevant readings for the thesis topic}. Approximated time: 10 minutes per article to ascertain usefulness. Ideally \textit{at least} 10 articles. 

2 \textbf{Developing research questions into surveys or an interview structure}. Approximated time: rough guess would be two weeks, depending on the format the questions are put into.

3 \textbf{Collect and analyse data from surveys and interviews}. Approximated time: typically for every hour in a recorded interview count on four hours transcribing. So even just 3 1 hour interviews would end up being an additional 12 hours of just \textit{writing down} what was verbally said (at least the useful bits).

4 \textbf{Format data and primary question into comprehensive article}. Approximated time: \textit{if} notes and job 3 have been well documented and articulated then the process could be a month if the article is Thesis-sized.

\textbf{
\subsection{Likely Pains:}
}

1 \textbf{Wasting time on finding relevant material}. 

2 \textbf{Articulating an academic question into survey format}. The challenge here will be one of translating. They need to be easy to engage with, require as little contextual knowledge to answer them as possible, and they need to be appropriately open or closed enough.

3 \textbf{Sifting through dialogue to find a pattern of discussion or a topic}. Most of the data in an interview is contextual and background to the real issue discussed. The articulation of the subject cannot be relied upon, and so it may be that only one useful piece of qualitative data is collected in an hour long interview.

4 \textbf{Background noise}. Often the setting of a recorded interview will have distracting background noise that is recorded along with the conversation. It can often drown out whatever is actually said and even if mindful of it the need to repeat information will take up unnecessary interview time.

5 \textbf{Complex Qualitative Data}. Often a question will be prepared with a quantitative answer in mind, and yet the answer received will have a qualitative aspect to it that requires the whole format to change.

\textbf{
\subsection{Pain Relievers:}
}
1 \textbf{Ask for help from a research assistant}. (At Macquarie they work in the library)

2 \textbf{Looking at past Research work to see how they have changed and articulated questions in surveys}. This will also give an indication of what kinds of questions are best used in what circumstances- in interviews as prompts or in surveys.

3 \textbf{When recording an interview, take a bell or some kind of audio prompt}. The purpose then is to use this prompt whenever something significant has been said in the interview itself. Later, when transcribing, these prompts can be used as audio markers to indicate when important information has come up.

4 \textbf{Arrange to meet in a quieter place}. If the informant knows that the interview is recorded it can be arranged so that the environment is suitable.

5 \textbf{Adopt a flexible, "flow" format for questions}. A little bit like a Kaban chart, propose a question to ask, see what the response is, regardless of whether it answers it or not, and then change the question according to the previous answer. This would work well between interviews to update them and potentially for a survey provided it was done more than once.

\textbf{
\subsection{Gains:}
}
1 \textbf{Get in contact with allocated MRes Research Assistants}.

2 \textbf{Look up past Thesis's and research case-studies}.

3 \textbf{Purchase a bike-bell}. (maybe something less intrusive...?)

4 \textbf{Find a consistent place to meet}.

5 \textbf{Buy stick-notes and a board}.

\textbf{
\subsection{Gain Creators:}
}
1 \textbf{Develop an ongoing relationship with research assistant for future projects}.

2. \textbf{Develop a familiarity with good/bad/better questions and question formats}.

3. \textbf{Introduce new ways to break up an interview to make the transcribing of it easier later on}.

4. \textbf{Create a stable environment for conducting interviews and improve on it with time and use}.

5. \textbf{Create a "stock" of questions that have proved useful based on their consistent appearance in the flow chart}.
\end{document}
